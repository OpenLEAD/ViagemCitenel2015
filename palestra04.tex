
\textbf{Desenvolvimento de Novos Coletores Solares de Condicionamento de Ar e
Refrigeração :}

O mercado crescente de ar condicionado propiciou um ambiente favoravel para o
aprimoramento de sistemas de condicionamento solares. A tecnologia de ar
condicionado solar foi dividido em duas, os de absorção e adsorção. Além deles,
existem os sistemas solares dessecantes que desumidificam o ar, atuando, assim,
 pelo uso do calor latente da água.
 O desenvolvimento do ar condicionado solar exposto na paletra foi pautado em
 simulações de volumes finitos feitos no software Ansys CFX, com criação de
 malhas através dos métodos \textit{sweep}, \textit{thin sweep} e
 \textit{inflation}. O sistema atingiu 74\% de eficiência na simulação, ficando
 ligeiramente abaixo no protótipo real.
