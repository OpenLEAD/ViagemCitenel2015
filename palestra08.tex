
\textbf{Visão computacional para monitoramento ambiental das áreas cobertas por
linhas de transmissão utilizando reconhecimento de padrões:}

O projeto consiste em um sistema formado por uma base e uma câmera fixa que
utiliza técnicas de reconhecimento de padrões para identificar focos de incêndio
no meio ambiente. O sistema faz análise de movimento, cor e
persistência de fumaças. O algoritmo é capaz de diferenciar fumaças de núvens
pela rapidez de movimento e gradiente de cor, mesmo utilizando câmeras de baixa
resolução. Um desafio técnico é a névoa, que pode atrapalhar o algoritmo de
identificação, portanto há um mecanismo de validação por usuários, já que os
resultados são disponibilizados online para todo o público, e um sistema
automático de aprendizado para refinar a técnica.
