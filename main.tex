%-----------------------------------------------------------------------
%
%   UFRJ  - Universidade Federal do Rio de Janeiro
%   COPPE - Coordenação dos Programas de Pós-graduação em Engenharia
%   PEE   - Programa de Engenharia Elétrica
%
%
%   Projeto EMMA - 
%
%                                                        10/jul/15, Rio
%                                                        Ramon R. Costa
%----------------------------------------------------------------------
%
%  Compilar usando PDFLaTeX
%
%----------------------------------------------------------------------
\documentclass[12pt,a4paper]{article}
\usepackage{macros/ROSApackages} 

\input{macros/ROSAsettings}
\input{macros/ROSAmacros}

%\def\PATH{file:c:/Users/Ramon/My Documents/projetos/2015/Projeto EMMA}

\begin{document}
%---------------------------------------------------------------------
\include{00_Frontpage}

%---------------------------------------------------------------------
%\tableofcontents

\newpage%
%---------------------------------------------------------------------
\section{Identificação}

\input{identificacao/identificacao}

\newpage%
%---------------------------------------------------------------------
\section{Sobre a visita}
Foram quatro dias de viagem à Costa do Sauípe, onde foi realizado o congresso
CITENEL e SEENEL:

Dia 16/08/2015:
\begin{itemize}
  \item Viagem e acomodação;
  \item Credenciamento no evento e planejamento de palestras técnicas;
\end{itemize}

Dia 17/08/2015:
\begin{itemize}
  \item Cerimônia de abertura e lançamento de revistas P\&D e EE da ANEEL;
  \item Palestra Magna - Prof. José Sidnei Colombo;
  \item Sessões técnicas;
\end{itemize}

Dia 18/08/2015:
\begin{itemize}
  \item Sessões técnicas;
\end{itemize}

Dia 19/08/2015:
\begin{itemize}
  \item Sessões técnicas;
  \item Viagem de volta;
\end{itemize}

%---------------------------------------------------------------------
\section{Considerações gerais}
A Agência Nacional de Enerngia Elétrica (ANEEL) entregou ao público a oitava
edição do Congresso de Inovação Tecnológica em Energia Elétrica (CITENEL) e a
quarta edição do Seminário de Eficiência Energética no Setor Elétrico (SEENEL),
realizados na Costa do Sauípe, Bahia, durante os dias 17, 18 e 19 de Agosto de
2015. O objetivo é ampliar a divulgação dos resultados alcançados nos
programas de Pesquisa e Desenvolvimento (P\&D) e de Eficiência Energética (EE)
regulados pela ANEEL.

A cerimônia de abertura e a palestra Magna abordaram os temas de inovação e
eficiência para o setor elétrico brasileiro, a importância da prática de
sustentabilidade, a preocupação com segurança do trabalho e primeiros socorros.

As sessões técnicas abordaram os seguintes temas: 1) redes inteligentes; 2)
planejamento de sistemas de energia elétrica; 3) geração termelétrica; 4)
eficiência energética; 5) fontes alternativas de geração de energia elétrica; 6)
supervisão, controle e proteção de sistemas de energia elétrica; 7) novos
materiais; 8) baixa renda; 9) qualidade e confiabilidade dos serviçõs de energia
elétrica; 10) medição, faturamento e combate a perdas comerciais; 11) operação
de sistemas de energia elétrica; 12) meio ambiente; 13) fontes alternativas de
geração de energia elétrica; comércio e serviços/industrial/fontes incentivadas
de energia elétrica; 14) educacional / residencial / iluminação pública; 15)
poder público; 16) segurança; 17) serviço público.

As sessões técnicas ocorriam concomitantemente, portanto foi elaborada uma
metodologia  selecionadas vinte palestras que 

\section{Planejamento de sessões técnicas}


\section{Sessões técnicas}


\section{Conclusões}


%---------------------------------------------------------------------
\end{document}
