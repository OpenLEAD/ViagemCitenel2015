
\textbf{Sistema de aeronaves não tripuladas multiplataforma de longo alcance
para inspeção de linhas de transmissão:}

Realizado por um grupo do ITA, este trabalho gerou 15 dissertações de
mestrado, 10 teses de doutorado e 15 trabalhos de conclusão de curso, 12 artigos internacionais e 12
artigos nacionais. O projeto consiste em dois veículos aéreos não tripulados
(VANTs) com autonomia de 3h a 4h, 25 kg a 50 kg de massa, e 7 kg a 14 kg de
carga (\textit{payload}). O VANT é catapultado e um piloto é responsável por
guiá-lo remotamente. O VANT sobrevoa próximo à linha de transmissão de energia
elétrica para inspeção. Foi realizado estudo de interferência eletromagnética da linha nos
diversos sensores que compõe o VANT e é feita uma análise de zona ótima para o
vôo, considerando a zona de exclusão devido à interferência e possibilidade de
colisão, e distância mínima necessária para inspeção por câmeras. A tecnologia
por hélices (Drones) mostraram resultados ruins devido à autonomia de apenas 15
minutos. A comunicação entre o VANT e o piloto foi amplificada por um balão, e
esse sistema de retransmissão foi um grande avanço técnico com pedido de
patente.
