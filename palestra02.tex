
\textbf{Robô de Inspeção de Linha -D311 :}

O palestrante apresentou o desenvolimento de um robô biomimético para inspeção
em linhas de transmissão de até 138kV. Possui sua estrutura inspirada em uma
lagarta de forma a ser capaz de atravessar obstáculos que apareçam.

O projeto foi feito com sua interface em ROS que é um framework livre. O
palestrante também ressaltou que o robô possuia avanços com relação a sua versão
anterior, como uma redução de 9kg pelo uso de titânio aeronáutico e formas ovais
para os elos, que minimizaram o ruído eletromagnético. 

Durante a sessão de perguntas o palestrante concluiu, explicando que a autonomia
do robô deverá ser expantida pelo uso de indução magnética, colhendo a energia
que passa pelo fio. Atualmente ele foi desenvolvido para ser introduzido nos
pontos de ancoragem de linha, porém o projeto para um robô de resgate já
foi desenvolvido pelo mesmo laboratório.
