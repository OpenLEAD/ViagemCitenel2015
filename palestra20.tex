\textbf{Concepção de um sistema para organização e processamento de imagens
multiespectrais, multitemporais e georeferenciadas de reservatórios para
monitoramento de bordas:}
Sistema de comparação de imagens amostradas em diferentes épocas, meses ou anos,
para monitoramento de reservatórios. O sistema é capaz de classificar pixels da
imagem em estruturas, construções, e outras formas. As imagens são adquiridas
via satélite, o que exige certo investimento, e o custo pode chegar a
R\$100.000,00 para duas a três coletas de imagem de média resolução por ano,
sendo que o comum é realizar aquisição a cda 2 meses, logo o monitoramento não é
em tempo real. O armazenamento das imagens é realizado em PotgreSQL e a
implementação do sistema é feita em PyQGIS (python). O desafio técnico consiste,
principlamente, na correção de angulação da imagem de satélite, a projeção
linear, e a manipulação de matrizes (imagem) para a diferenciação e
classificação.
