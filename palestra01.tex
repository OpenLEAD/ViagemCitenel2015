\textbf{Desenvolvimento de veículo dubaquático para inspeção de usinas
hidrelétricas:}
Desenvolvimento de um ROV de dimensões 400x400x600 mm de fibra de vidro com
câmeras e SONAR, e um sistema de navegação com algoritmo de Monte
Carlo. O robô assemelha-se muito ao ROV da empresa Seabotix em sua composição
mecânica, designer, sensores e distribuição de atuadores. Como o investimento
não foi alto, os atuadores não são comerciais, mas sim projetados em Solidworks
(mecânica) e Proteus (eletrônica), e montados pela própria USP, sendo capaz de
suportar correntezas de 2 m/s. O robô foi testado em usinas
hidrelétricas e foram observado ruídos eletromagnéticos na transmissão de
imagens, sendo este o desafio para a continuação do projeto.
