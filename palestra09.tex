
\textbf{Rede de Sensores Passivos para Medição de Integridade de Equipamentos em
Sistemas de Energia com Transmissão sem Fio :}

A rede de sensores foi utilizada para medir temperatura de disjuntores e buchas,
porém poderiam ter sido utilizados outros sensores. Os sensores eram baseados em
tecnologia SAW (Surface Acoustic Wave) constituido por um cristal piezoelétrico
que gera uma resposta final similar ao RFID.

A rede foi contruída sobre um protocolo ZigBee para compatibilizar o baixo
consumo como resiliência, visto que ZigBee suporta \textit{full mesh}. Os
roteadores ZigBee possuiam capacidade de transmitir ao nó central em RS232 ou
fibra otica, onde foi escolhido a ultima devido a problemas com interferência
eletromagnética.

O desafio técnico enfrentado foi a criação de um hardware baseado em FPGA para
sincronizar a resposta dos sensores e evitar colisões que degradassem a
comunicação. Além dos sensores de temperatura, ao final, o palestrante também
comentou o uso de sensores de deformação.
