
\textbf{Instalação de Gerador Solar Fotovoltaico na Aeronáutica Conenectado á
Rede Elétrica do Sistema Isolado de Fernando de Noronha:}

A ilha de Fernando de Noronha possuía para seu abastecimento energético apenas
uma usina a disel de 2MW e, por possuir capacidade de estocagem limitada, está
sempre dependende do alto tráfego de embarcações para manter a usina abastecida.

Ao custo de um investimento de R\$3,9 milhões foi instalada, na base da
Aeronáutica, uma usina fotovoltáica de 402,78kW. Esta é responsável, agora, por
5\% da energia consumida na ilha, chegando ao pico de 20\% do consumo
instantâneo da ilha por volta de meio dia.
Com uma geração anual de 620kWh, a usina fotovoltáica supre 70\% da demanda
energética da Aeronáutica e o restante é distribuido pela rede, sem armazenagem.
Todo esse equipamento recebe manutenção a cada 15 dias devido à alta
concentração de sal no ambiente.
